\documentclass[11pt]{article}

\usepackage{fullpage,times,hyperref,graphicx}

% The next line is the title of your paper. Update it.
\title{The Title of Your Paper} 
% The next line is the author information for your paper. Update it.
\author{Your full name and UT EID}
% The next line is the timestamp for your paper. DO NOT CHANGE IT.
\date{\today}

% The next line controls the formatting of your references. DO NOT CHANGE IT.
\bibliographystyle{plain} 

\begin{document}
\maketitle

\section{Introduction}
This is a \LaTeX\ template file for you to use in your writing assignments in C S 429.
The text in this file shows you how to use several of the features of \LaTeX.
It may help you to think of \LaTeX\ as a powerful markup language for documents, 
similar to HTML or XML.

\emph{Every assignment in C S 429 is an individual assignment.}
Before you begin, 
please take the time to review the course policy on academic integrity at:
\url{https://www.cs.utexas.edu/academics/conduct}.

\section{Details of the assignment}
The word \emph{system} is among the most overloaded in the English language.
Oxford lists \href{https://www.oxfordlearnersdictionaries.com/us/definition/american_english/system}{four senses} of the word in American English 
and \href{https://www.oxfordlearnersdictionaries.com/us/definition/english/system}{five} in British English;
Merriam-Webster \href{https://www.merriam-webster.com/dictionary/system}{lists} five major definitions,
with multiple finer shades of meaning within them;
Chambers \href{https://chambers.co.uk/search/?query=system&title=21st}{lists} nine;
and Collins \href{https://www.collinsdictionary.com/dictionary/english/system}{lists} even more.
This multiplicity of connotations begs the question of exactly what constitutes a system in the field of computer science.

You are to write an essay of no more than 1000 content words~\footnote{That is, excluding title, references, and the like.}
 that accomplishes the following tasks:
\begin{itemize}
\item surveys a selection of definitions of the term;
\item synthesizes your own definition for it that is relevant to the kinds of systems we study in C S 429 and C S 439; and
\item creates a rubric for fully describing a system.
\end{itemize}
Of course,
you need to convince your reader
(me)
of the soundness of your arguments all through the essay.

To get you started,
I have provided an annotated list of readings drawn from sources spanning multiple disciplines.
PDF versions of the materials,
where applicable,
have been uploaded to Canvas.
\begin{itemize}
\item 
Ackoff's paper~\cite{ackoff:1994} provides a good framework for systems.
Chapter~2 from Boardman and Sauser's book~\cite{boardman:sauser:2008} defines systems from a fairly broad perspective.
The authors come from an engineering and management background.
\item
The ``Swiss cheese'' paper by Reason~\cite{reason:2000} is a classic in the systems community.
Written by a psychologist,
it was referenced often last year in the early days of the pandemic.
A recent episode of The Ezra Klein Show podcast featuring Zeynep Tufecki~\cite{klein:2021} also focuses on systems thinking 
in the context of sociology and medicine.
\item
The biologist {v}on Bertalanffy was a major driving force behind what we today call systems biology.
His 1950 paper~\cite{vonBertalanffy:1950} is focused on the notion of systems in the natural sciences.
\item
Coming to computer science,
Parnas's paper on mission-critical systems software~\cite{parnas:1985} contains,
among other things,
a taxonomy of systems.
Simon's paper on the architecture of complexity~\cite{simon:1962} is an outstanding paper by a pioneer and a giant in our field, 
who won both the A. M. Turing Award (1975) and the Nobel Memorial Prize in Economic Sciences (1978).
But probably the two most famous papers on system design are the ones written ten years apart by Butler Lampson.
The first paper~\cite{lampson:1983},
from 1983,
distills Lampson's experiences from designing and implementing a large array of computer systems.
The second paper~\cite{lampson:1992},
his Turing Award lecture,
is a rumination on how ``depressingly little'' has been learned in the decade since his earlier paper.
\end{itemize}

\emph{You are required to meaningfully incorporate at least three of these readings in your paper.}
You are free to use additional materials if you so choose.

I may update this list with additional readings.
Such updates will be announced on Piazza.

Needless to say,
the paper is expected to be in grammatical English and in an appropriate academic style.
Come talk to me if you are unsure of what this entails.
If you need help with the mechanics of writing,
your best resource is \href{http://uwc.utexas.edu/}{The University Writing Center}.
They have a wide variety of facilities,
including one-on-one consultations with over 100 consultants,
to help you.

You must use \LaTeX\ to format your paper.
I will provide the necessary document templates and a makefile to simplify the workflow.
The UTCS lab machines all have the \LaTeX\ software tools installed on them.

\section{Handin}
You will provide a link to the PDF document generated by \LaTeX: 
once at the time of the checkpoint,
and then again at the time of final submission.
Generate the PDF document by following the instructions in the Makefile that accompanies this template.

The first checkpoint should be a first draft.
I will read your draft and provide you feedback on it,
with the expectation that you will incorporate the feedback into your final version.

\section{Evaluation}
The journal is worth 10\% of the total course grade.
The first checkpoint counts for three points,
and the final version of the paper counts for seven points.
Detailed grading rubrics will be provided on Canvas.

% The following line says where to find your bibliography database.
\bibliography{term-paper-template}
\end{document}